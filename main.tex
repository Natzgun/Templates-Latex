\documentclass{practicaEPCC}
\usepackage{hyperref}
\usepackage{lmodern}
\school{Escuela Profesional de Ciencia de la Computación}
\course{Análisis y Diseño de Algoritmos}
\professor{Alfredo Paz Valderrama}
\title{Tarea de Notación Asintótica y Recurrencias}
\author{Erick Malcoaccha y Erik Ramos}
\date{\today}
\begin{document}
\maketitle
\begin{abstract}
Los siguientes ejercicios fueron resueltos por los autores a partir de la \href{https://users.dcc.uchile.cl/~gnavarro/cc40a/guia1.ps.gz}{guía 1} del curso de Algoritmos del Profesor Gonzalo Navarro de la Universidad de Chile.
    
\end{abstract}
\section{Ordenando Funciones}
Para los ordenamientos deberá comparar las funciones usando límites o sustitución. Si no incluye los métodos y operaciones que usó para encontrar las respuestas, no tendrá puntos.
\begin{enumerate}
  \item Ordene las funciones en orden $O()$ creciente, indicando los grupos que tienen el mismo orden.

    Para resolver este primer punto vamos a usar limites hacia el infinito
    \item Haga los mismo con las siguientes funciones

\end{enumerate}

\section{Manipulación de $O()$}
Se puede definir $O(g(n))$ como el conjunto de todas las funciones que son $O(g(n))$. Así, ``$f(n)$ es $O(g(n))$'' se puede reescribir como $f(n) \in O(g(n))$. Abusando de la notación se dice $f(n)=O(g(n))$.
En los siguientes ejercicios, sea $e(n) = o(1)$, $f_i(n) = O(g_i(n))$ y $f_i(n) = \Omega(h_i(n))$. Pruebe que:

\begin{enumerate}
    \item $O(O(f(n))) = O(f(n))$
\end{enumerate}

\section{Recurrencias y Funciones Generatrices}
Resuelva las siguientes recurrencias sin usar y usando funciones generatrices. La solución debe ser exacta para infinitos $n$ (diga cuales).

\begin{enumerate}
    \item $T(n) = T(n - 1) + n - 1$, $T(1) = 2$
\end{enumerate}

\end{document}
